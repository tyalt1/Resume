\documentclass[11pt, a4paper]{article} %Sets the default text size to 11pt and class to article.
%-------------------------------------------------------------------------------
\topmargin=0.0in %length of margin at the top of the page (1 inch added by default)
\oddsidemargin=0.0in %length of margin on sides for odd pages
\evensidemargin=0in %length of margin on sides for even pages
\textwidth=6.5in %How wide you want your text to be
\marginparwidth=0.5in
\headheight=0pt %1in margins at top and bottom (1 inch is added to this value by default)
\headsep=0pt %Increase to increase white space in between headers and the top of the page
\textheight=9.0in %How tall the text body is allowed to be on each page
\pagenumbering{gobble} %Remove page numbers

\newcommand{\tabitem}{~~\llap{\textbullet}~~}

\begin{document}
%Name
\centerline{\Huge \sc Tyler Alterio}

%Contact Information
\centerline{1 Rockland Terrace - Natick, MA \textbullet \hspace{1pt} (508)-314-9756 \textbullet \hspace{1pt} Tyler\_{}Alterio@student.uml.edu \textbullet \hspace{1pt} GitHub ID: tyalt1}

\noindent \line(1,0){470}\\

%Education
\noindent {\Large \bf Education}
\smallskip \\
University of Massachusetts Lowell - Anticipated Graduation Fall 2016 \\
B.S. Electrical Engineering and B.S. Computer Science \\
GPA: 3.4 \hspace{75pt} Dean's List All Semesters

\noindent \line(1,0){470}\\

%Professional Work Experience
\noindent {\Large \bf Professional Work Experience}
\smallskip \\
\centerline{\bf University of Massachusetts \textbullet \hspace{1pt} Lowell, MA \hfill August 2015 to Present}
\leftline{\bf Research Assistant}
\begin{itemize}
\itemsep0em
	\item Experimented with P4, an open-source domain-specific network description language.
	\item Develop method to compile P4 behavioral model into CUDA implementation.
\end{itemize}


\centerline{\bf Philips Color Kinetics \textbullet \hspace{1pt} Burlington, MA \hfill June 2014 to January 2015}
\leftline{\bf Firmware Engineer Co-Op}
\begin{itemize}
\itemsep0em
	\item Practical use of Git version control and tools. In addition to Github management.
	\item Implemented new features in firmware of PIC based devices with MPLAB IDE.
	\item Created command-line utility in Qt framework to interface with fixtures via UDP.
	\item Worked with team of 8 software engineers, utilizing Agile Scrum workflow.
\end{itemize}

\noindent \line(1,0){470}\\

%Skills
\noindent {\Large \bf Skills} \smallskip
\vspace{-10pt}
\begin{center}
\begin{tabular}{l|l}
	{\bf Languages} & {\bf Software} \\
	\tabitem \textit{General:} C, C++ & \tabitem \textit{Operating Systems:} Windows (up to 10), Linux (Debian-based)\\
	\tabitem \textit{Object-Oriented:} Java & \tabitem \textit{IDEs:} Visual Studio, Qt Creator, IntelliJ, PyCharm\\
	\tabitem \textit{Scripting:} Python, Perl & \tabitem \textit{Version Control:} Git\\
	\tabitem \textit{Shell:} Bash, PowerShell & \tabitem \textit{Hardware:} Arduino, Raspberry Pi\\
	\tabitem \textit{Functional:} Clojure & \tabitem \textit{Lab Equipment:} Oscilloscope, Multimeter,\\
	\tabitem \textit{Numerical:} MATLAB & ~\\
\end{tabular}
\end{center}

\noindent \line(1,0){470}\\

%Relevant Course Work
\noindent {\Large \bf Relevant Course Work}
\begin{itemize}
\itemsep0em
	\item \underline{Intro to Engineering 2:} MATLAB application and use of BASIC STAMP microcontrollers.	
	\item \underline{Computing 4:} Adoption of OpenCV computer vision library and Qt libraries for C++.
	\item \underline{Electronics 1:} Application of operational amplifiers, diodes, MOSFETs, and BJTs.
	\item \underline{Microprocessor System Design:} Program design in x86 and PIC assembly.
\end{itemize}

%End
%\vfill
\noindent \line(1,0){470} \\

\end{document}
